\documentclass[12pt]{beamer}\usepackage[]{graphicx}\usepackage[]{color}
%% maxwidth is the original width if it is less than linewidth
%% otherwise use linewidth (to make sure the graphics do not exceed the margin)
\makeatletter
\def\maxwidth{ %
  \ifdim\Gin@nat@width>\linewidth
    \linewidth
  \else
    \Gin@nat@width
  \fi
}
\makeatother

\definecolor{fgcolor}{rgb}{0.345, 0.345, 0.345}
\newcommand{\hlnum}[1]{\textcolor[rgb]{0.686,0.059,0.569}{#1}}%
\newcommand{\hlstr}[1]{\textcolor[rgb]{0.192,0.494,0.8}{#1}}%
\newcommand{\hlcom}[1]{\textcolor[rgb]{0.678,0.584,0.686}{\textit{#1}}}%
\newcommand{\hlopt}[1]{\textcolor[rgb]{0,0,0}{#1}}%
\newcommand{\hlstd}[1]{\textcolor[rgb]{0.345,0.345,0.345}{#1}}%
\newcommand{\hlkwa}[1]{\textcolor[rgb]{0.161,0.373,0.58}{\textbf{#1}}}%
\newcommand{\hlkwb}[1]{\textcolor[rgb]{0.69,0.353,0.396}{#1}}%
\newcommand{\hlkwc}[1]{\textcolor[rgb]{0.333,0.667,0.333}{#1}}%
\newcommand{\hlkwd}[1]{\textcolor[rgb]{0.737,0.353,0.396}{\textbf{#1}}}%
\let\hlipl\hlkwb

\usepackage{framed}
\makeatletter
\newenvironment{kframe}{%
 \def\at@end@of@kframe{}%
 \ifinner\ifhmode%
  \def\at@end@of@kframe{\end{minipage}}%
  \begin{minipage}{\columnwidth}%
 \fi\fi%
 \def\FrameCommand##1{\hskip\@totalleftmargin \hskip-\fboxsep
 \colorbox{shadecolor}{##1}\hskip-\fboxsep
     % There is no \\@totalrightmargin, so:
     \hskip-\linewidth \hskip-\@totalleftmargin \hskip\columnwidth}%
 \MakeFramed {\advance\hsize-\width
   \@totalleftmargin\z@ \linewidth\hsize
   \@setminipage}}%
 {\par\unskip\endMakeFramed%
 \at@end@of@kframe}
\makeatother

\definecolor{shadecolor}{rgb}{.97, .97, .97}
\definecolor{messagecolor}{rgb}{0, 0, 0}
\definecolor{warningcolor}{rgb}{1, 0, 1}
\definecolor{errorcolor}{rgb}{1, 0, 0}
\newenvironment{knitrout}{}{} % an empty environment to be redefined in TeX

\usepackage{alltt}
\usepackage{graphicx}
\usepackage{tikz}
\setbeameroption{hide notes}
\setbeamertemplate{note page}[plain]
\usepackage{listings}

\input{../header.tex}

%------------------------------------------------
% end of header
%------------------------------------------------

\title{Command Line - Part 1}
\subtitle{STAT 133}
\author{\href{http://www.gastonsanchez.com}{Gaston Sanchez}}
\institute{\href{https://github.com/ucb-stat133/stat133-fall-2016}{\tt \scriptsize \color{foreground} github.com/ucb-stat133/stat133-fall-2016}}
\date{}
\IfFileExists{upquote.sty}{\usepackage{upquote}}{}
\begin{document}


{
  \setbeamertemplate{footline}{} % no page number here
  \frame{
    \titlepage
  } 
}

%------------------------------------------------

\begin{frame}
\begin{center}
\Huge{\hilit{GUIs}}
\end{center}
\end{frame}

%------------------------------------------------

\begin{frame}
\frametitle{Graphical User Interfaces}

\bi
  \item Windows and Mac use a Graphical User Interface (GUI) for you to interact with the OS.
  \item GUIs are easy to learn
  \item GUIs rely on visual displays
  \item GUIs can be extremely useful
  \item GUIs have improved the friendliness and usability of computers
\ei

\end{frame}

%------------------------------------------------

\begin{frame}
\frametitle{GUIs or Command Line?}

\bi
  \item However, GUIs come with trade-offs
  \item They don't allow you to have more control over what your computer can do
  \item Some operations are labor intensive and repetitive
  \item You organize things by clicking and dragging with the cursor (which reduces reproducibility)
\ei

\end{frame}

%------------------------------------------------

\begin{frame}
\frametitle{GUI Disadvantages}
\bbi
  \item Lack of repeatability
  \item Lack of reproducibility
  \item Some tasks may be labor intensive using a GUI
  \item GUIs limit analyses on a cluster of computers
\ei
\end{frame}

%------------------------------------------------

\begin{frame}
\begin{center}
\Huge{\hilit{Command Line}}
\end{center}
\end{frame}

%------------------------------------------------

\begin{frame}
\frametitle{Command Line}
\bbi
  \item Instead of using a GUI, we can use a command line program
  \item The command line program is known as the \textbf{shell}
  \item By typing commands we perform tasks on the computer (without using a mouse)
\ei

\end{frame}

%------------------------------------------------

\begin{frame}
\frametitle{Shell}

\bi
  \item You're working with a program called the \textbf{shell}
  \item The shell interprets the commands you enter
  \item It runs the program you've asked for
  \item It coordinates what happens between you and the operating system
  \item There are various kinds or flavors of shells: e.g. Bourne (BASH), Korn, C shell
\ei

\end{frame}

%------------------------------------------------

\begin{frame}
\frametitle{Command Line}

\bbi
  \item To interact with the shell we need a \textbf{terminal emulator}
  \item In Unix-like systems (e.g. Mac) the terminal is usually known as ``terminal''
  \item Windows does not really provide a terminal; instead it provides the \textit{command prompt}
\ei

\end{frame}

%------------------------------------------------

\begin{frame}
\frametitle{Command Prompt in Windows}

\bb{Finding MS Windows command prompt}
\bi
  \item Click the \textbf{Start} button
  \item Click \textbf{All Programs}
  \item Click \textbf{Accessories}
  \item Click \textbf{Command Prompt}
\ei
Windows command prompt is not a UNIX shell
\eb

\end{frame}

%------------------------------------------------

\begin{frame}
\frametitle{Shells for Windows}

\bi
  \item Instead of using the command prompt you can use ad-hoc shell environments for Windows
  \item e.g. Git-Bash, PowerShell, Cygwin
  \item Git for Windows provides a BASH emulation
  \item PowerShell is part of Windows Management Framework 4.0
  \item Cygwin is large collection of GNU and Open Source tools
\ei

\end{frame}

%------------------------------------------------

\begin{frame}
\frametitle{Mac Terminal}
\begin{center}
\ig[width=3cm]{images/terminal.png}
\end{center}

\bi
  \item Go to \textbf{Applications}
  \item Go to \textbf{Utilities}
  \item Click \textbf{Terminal}
\ei
\end{frame}

%------------------------------------------------

\begin{frame}
\frametitle{Try Some Commands}
\bi
  \item \code{date} (current time and date)
  \item \code{cal} (calendar of current month)
  \item \code{df} (amount of free space in your disk drives)
  \item \code{who} (logged in users)
  \item \code{echo `Hello'}
\ei
\end{frame}

%------------------------------------------------

\begin{frame}
\frametitle{Shell}
\bi
  \item Shells run in terminal emulators, or \textbf{terminals}
  \item In Mac OS X, the default reminal program is called \textbf{Terminal}
  \item The command line is displayed within the terminal window
  \item The program behind the terminal is the \textbf{shell}
  \item There are many different shell programs
\ei
\end{frame}

%------------------------------------------------

\begin{frame}
\frametitle{BASH}

The most common type of shell is BASH
\bi
  \item BASH: Bourne Again SHell
  \item BASH is the default shell for Linux
  \item BASH is usually the default shell on Mac
  \item type {\hilit \code{echo \$SHELL}} to see your shell
  \item type {\hilit \code{bash}} to get a bash shell
\ei
\end{frame}

%------------------------------------------------

\begin{frame}
\frametitle{BASH}
\bi
  \item A shell does much more than simply run commands
  \item It has wildcards for matching filenames
  \item It has a command history to recall previous commands quickly
  \item It has pipes for making the output of one command become the input of another
  \item It has variables for storing values for use by the shell
\ei
\end{frame}

%------------------------------------------------

\begin{frame}
\frametitle{Command \code{who}}
\bbi
  \item \code{who} displays a list of users that are currently logged in
  \item \code{who am i} (\code{whoami}) tells you the current user name
\ei
\end{frame}

%------------------------------------------------

\begin{frame}
\frametitle{Shell Command Syntax}

\code{command -options arg1 arg2}

\bigskip
\bi
  \item Blanks and \code{"-"} are delimiters
  \item The number of arguments may vary
  \item An argument comes at the end of the command line
  \item It's usually the name of a file or some text
  \item Many commands have default arguments
\ei
\end{frame}

%------------------------------------------------

\begin{frame}
\frametitle{Date and Calendar}
\bi
  \item \code{date}
  \item \code{cal} (current calendar year)
  \item \code{cal july 2015} (July 2015)
  \item \code{cal jan 2000}
  \item \code{ncal -w july 2015} (week number)
\ei
\end{frame}

%------------------------------------------------

\begin{frame}
\frametitle{Options}

\code{command -options arg1 arg2}

\bigskip
\bi
  \item Options come between the command and the arguments
  \item They tell the command to do something other than its default
  \item They are usually prefaced with one or two hyphens
  \item e.g. \code{ncal -w july 2015}
\ei

\end{frame}

%------------------------------------------------

\begin{frame}
\frametitle{Some Control Sequences}

\begin{center}
 \begin{tabular}{l l}
  \hline
   keys & description \\
  \hline
  \code{Ctrl + l} & clear screen \\  
  \code{Ctrl + c} & stop current command \\
  \code{Ctrl + z} & suspend current command \\
  \code{Ctrl + k} & kill to end of line \\
  \code{Ctrl + r} & search history \\
  \code{Ctrl + n} & next history item \\
  \code{Ctrl + p} & previous history item \\
  \hline
 \end{tabular}
\end{center}

\end{frame}

%------------------------------------------------

\begin{frame}
\frametitle{Manual Documentation}

\bi
  \item To see the help documentation of a command use {\hilit \code{man}} followed by the name of the command:
  \bi
    \item \code{man cal}
    \item \code{man date}
    \item \code{man who}
  \ei
  \item {\hilit \code{q}} quits manual documentation
\ei
\end{frame}

%------------------------------------------------

\begin{frame}
\frametitle{Logging Out}
\bi
  \item {\hilit \code{exit}} logs you out
  \item {\hilit \code{q}} quits manual documentation
\ei
\end{frame}

%------------------------------------------------

\begin{frame}
\begin{center}
\Huge{\hilit{System Navigation}}
\end{center}
\end{frame}

%------------------------------------------------

\begin{frame}
\frametitle{Filesystem Reminder}
\bi
  \item The nested hierarchy of folders and files on your computer is called the \textbf{filesystem}
  \item The filesystem follows a tree-like structure
  \item The root directory is the most includive folder on the system
  \item The root directory serves as the container ofr all other files and folders
  \item A Unix-based system (e.g. OS X) has a single root directoyr
  \item Windows users usually have multiple roots (\code{C:, D:}, etc)
\ei
\end{frame}

%------------------------------------------------

\begin{frame}
\frametitle{Paths}
\bi
  \item Each file and directory has a unique name in the filesystem
  \item Such unique name is called a \textbf{path}
  \item A path can be \textbf{absolute} or \textbf{relative}
  \item An \textbf{absolute path} is a complete and unambiguous description of where something is in relation to the root
  \item A \textbf{relative} describes where a folder or file is in relation to another folder
\ei
\end{frame}

%------------------------------------------------

\begin{frame}
\frametitle{Paths}
\bi
  \item There are two special relative paths: \code{.} and \code{..}
  \item The single period \code{.} refers to your current directory
  \item The two periods means your parent directory, one level above
\ei
\end{frame}

%------------------------------------------------

\begin{frame}
\frametitle{Home Directory}
\bi
  \item User's personal files are found in the \code{/Users} directory
  \item A user directory is the \textbf{home} directory
  \item {\hilit \code{cd}} (with no other arguments) returns you to your home directory
  \item {\hilit \code{echo \$HOME}} prints your home directory
  \item {\hilit \code{cd $\sim$}} takes you to your home directory
\ei
\end{frame}

%------------------------------------------------

\begin{frame}
\frametitle{Working Directory}
\bi
  \item Another special type of directory is the so-called \textbf{working directory}
  \item The working directory is the current directory where you perform any task
  \item {\hilit \code{pwd}} prints the working directory
\ei
\end{frame}

%------------------------------------------------

\begin{frame}
\frametitle{Changing Directories}
\bi
  \item \code{cd}
  \item \code{cd ..}
  \item \code{cd /}
  \item \code{cd $\sim$}
  \item \code{cd $\sim$/Documents}
\ei
\end{frame}

%------------------------------------------------

\begin{frame}[fragile]
\frametitle{Absolute Path Names}
\begin{center}
\ig[width=5cm]{images/treefiledir.pdf}
\end{center}

From the root directory to \code{D}: \\
{\hilit \code{cd /A/B/D}}

\end{frame}

%------------------------------------------------

\begin{frame}[fragile]
\frametitle{Relative Path Names}
\begin{center}
\ig[width=5cm]{images/treefiledir.pdf}
\end{center}

Changing directories from \code{D} to \code{C} \\
{\hilit \code{cd ../../C}}

\end{frame}

%------------------------------------------------

\begin{frame}
\frametitle{Listing Contents in a Directory}
\bi
  \item \code{ls}
  \item \code{ls -1} (one entry per line)
  \item \code{ls -l} (list in long format)
  \item \code{ls -a} (show files starting with a dot)
  \item \code{man ls} (manual documentation)
\ei
\end{frame}

%------------------------------------------------

\begin{frame}
\frametitle{Listing Contents in a Directory}
\bbi
  \item \code{ls /} (specify root directory)
  \item \code{ls /usr} (specifying a directory)
  \item \code{ls $\sim$} (home directory)
  \item \code{ls -lt} (long format, sorted by modification time)
\ei
\end{frame}

%------------------------------------------------

\begin{frame}[fragile]
\frametitle{Listing Contents}
\begin{center}
\ig[width=5cm]{images/treefiledir.pdf}
\end{center}

Show contents in D from C \\
{\hilit \code{ls ../B/D/}}

\end{frame}

%------------------------------------------------

\begin{frame}
\begin{center}
\Huge{\hilit{Inspecting Files}}
\end{center}
\end{frame}

%------------------------------------------------

\begin{frame}
\frametitle{File Permissions}
\bi
  \item run the command: \code{ls -l}
  \item directories may be displayed as: \code{drwxr-xr-x}
  \item files may be displayed as: \code{-rw-r--r--}
  \item file permissions are the 10 most left characters
  \item \code{r} means reads
  \item \code{w} means write
  \item \code{x} means execute
\ei
\end{frame}

%------------------------------------------------

\begin{frame}
\frametitle{File Permissions}
Read from left to right the permissions mean

\begin{center}
 \begin{tabular}{c l}
  \hline
   position & description \\
  \hline
  1 & File type. A dash \code{-} means a plain file \\
    & and \code{d} means a directory. \\
    & There are other less common options. \\  
  2-4 & Owner permissions: read, write, and execute \\
      & permissions for the file's owner. \\  
  5-7 & Group permissions: read, write, and execute \\
      & permissions for the file's group. \\
  8-10 & World permissions: read, write, and execute \\
       & permissions for all other users. \\
  \hline
 \end{tabular}
\end{center}
\end{frame}

%------------------------------------------------

\begin{frame}
\frametitle{Type of File}
Determine the type of a file: \\
\code{file} \textit{filename}
\end{frame}

%------------------------------------------------

\begin{frame}
\frametitle{Some commands for inspecting text files}
\bi
  \item \code{wc} \textit{filename}
  \item \code{cat} \textit{filename}
  \item \code{head} \textit{filename}
  \item \code{tail} \textit{filename}
  \item \code{more} \textit{filename}
  \item \code{less} \textit{filename}
\ei
\end{frame}

%------------------------------------------------

\begin{frame}
\frametitle{Viewing file contents with \code{less}}
\bi
  \item There are several commands that display the contents of text files
  \item The most commonly used file viewer is {\hilit \code{less}}
  \item \code{less} presents the contents of that file on the screen one page at a time
  \item There are various keyboard surtcuts to navigate in \code{less}
\ei
\end{frame}

%------------------------------------------------

\begin{frame}
\frametitle{Viewing file contents with \code{less}}
{\small
\begin{center}
 \begin{tabular}{l l}
  \hline
   key & description \\
  \hline
  Page Up or \textbf{b} & scroll back one page \\
  Page Down or & scroll forward one page \\
  \textbf{space} & \\
  Up Arrow & scroll up one line \\  
  Down Arrow & scroll down one line \\
  \textbf{G} & move to the end of text file \\  
  \textbf{1G} or \textbf{g} & move to the beginning of the text file \\
  \textbf{/hello} & search forward to next occurrence of hello \\
  \textbf{n} & search for the next search occurrence \\
  \textbf{h} & display help screen \\
  \textbf{q} & quit less \\
  \hline
 \end{tabular}
\end{center}
}
\end{frame}

%------------------------------------------------

\begin{frame}
\frametitle{Quoting Files}
If you want a word to contain whitespace (e.g. a filename with a space in it), surround it with single or double quotes to make the shell treat it as a unit: \\
\code{ls "My file"}
\end{frame}

%------------------------------------------------

\begin{frame}
\frametitle{Exploring a file}
\bi
  \item \code{cd} into a given directory
  \item List the directory contents with \code{ls -l}
  \item Determine the contents of a file with \code{file}
  \item If it looks like it might be text, try viewing it with \code{less}
\ei
\end{frame}

%------------------------------------------------

\begin{frame}
\frametitle{Editing text files at the command line}
\bi
  \item Sometimes it is more convenient to create or modify a file right at the command line
  \item Although \code{less} is a convenient file viewer, it does not allow you to edit the contents
  \item Depending on your operating system and shell tool, you may have one or more command-line text editors:
  \item e.g. \code{vi}, \code{nano}, \code{gedit}
\ei
\end{frame}

%------------------------------------------------

\begin{frame}
\frametitle{Editing text files at the command line}
\bi
  \item One common text editor is \code{vi} (there's also \code{vim})
  \item It should be available in Mac, and also in Git-Bash (Windows)
  \item Depending on your operating system and shell tool, you may have one or more command-line text editors:
  \item Type {\hilit \code{which vi}} to fing out if you have it
\ei
\end{frame}

%------------------------------------------------

\begin{frame}
\frametitle{Editing text files with \code{vi}}
\bi
  \item To create and start editing a file simply type \code{vi} followed by the name of the new file:
  \item[] {\hilit \code{vi newfile.txt}}
  \item Press the \textbf{I} key to start editing content
  \item When you're done, press the \textbf{ESC} key
  \item Then type \code{:wq} to save and quit
  \item You can reopen it again with: \code{vi newfile.txt}
\ei

Google \textbf{vi cheat sheet} to find more information
\end{frame}

%------------------------------------------------

\begin{frame}
\begin{center}
\Huge{\hilit{File Management}}
\end{center}
\end{frame}

%------------------------------------------------

\begin{frame}
\frametitle{Managing Files}
Common actions
\bi
  \item creating a directory
  \item creating a file
  \item copying a file
  \item moving a file
  \item deleting a file
  \item searching a file
\ei
\end{frame}

%------------------------------------------------

\begin{frame}
\frametitle{Managing Files}
Common actions
\bi
  \item creating a directory: \code{mkdir}
  \item creating a file: usually through a text editor
  \item copying a file: \code{cp}
  \item moving a file: \code{mv}
  \item deleting a file: \code{rm}
  \item searching a file: ?
\ei
\end{frame}

%------------------------------------------------

\begin{frame}[fragile]
\frametitle{Creating Directories and Files}

Create a directory \code{"summer2015"} in my Documents
\begin{knitrout}\footnotesize
\definecolor{shadecolor}{rgb}{0.969, 0.969, 0.969}\color{fgcolor}\begin{kframe}
\begin{alltt}
cd ~/Documents
mkdir summer2015
\end{alltt}
\end{kframe}
\end{knitrout}

\bigskip
Create an empty file \code{"README.md"} in \code{summer2015}
\begin{knitrout}\footnotesize
\definecolor{shadecolor}{rgb}{0.969, 0.969, 0.969}\color{fgcolor}\begin{kframe}
\begin{alltt}
cd summer2015
touch README.md
\end{alltt}
\end{kframe}
\end{knitrout}

\end{frame}

%------------------------------------------------

\begin{frame}
\frametitle{Copying Files}
\bi
  \item \code{cp} is the command to copy files
  \item \code{cp} can be used in two ways:
  \item \code{cp file1 file2} copies \code{file1} into \code{file2}
  \item \code{cp file1 directory} copies \code{file1} into a directory (directory must already exists)
\ei
\end{frame}

%------------------------------------------------

\begin{frame}[fragile]
\frametitle{Copying Files}

Copying \code{functions.R} from Documents to HW6
\begin{knitrout}\footnotesize
\definecolor{shadecolor}{rgb}{0.969, 0.969, 0.969}\color{fgcolor}\begin{kframe}
\begin{alltt}
cp ~/Documents/functions.R ~/Desktop/HW6/
\end{alltt}
\end{kframe}
\end{knitrout}

\bigskip
Copying \code{starwars.csv} to current directory
\begin{knitrout}\footnotesize
\definecolor{shadecolor}{rgb}{0.969, 0.969, 0.969}\color{fgcolor}\begin{kframe}
\begin{alltt}
cp ~/Documents/starwars.csv .
\end{alltt}
\end{kframe}
\end{knitrout}

\end{frame}

%------------------------------------------------

\begin{frame}[fragile]
\frametitle{Deleting files}

Deleting \code{README.md} and \code{starwars2.csv}
\begin{knitrout}\footnotesize
\definecolor{shadecolor}{rgb}{0.969, 0.969, 0.969}\color{fgcolor}\begin{kframe}
\begin{alltt}
cd ~/Documents/summer2015
rm README.md
rm starwars2.csv
\end{alltt}
\end{kframe}
\end{knitrout}

\end{frame}

%------------------------------------------------

\begin{frame}
\frametitle{Wildcards}

\bi
  \item the shell provides special characters to specify filenames
  \item these special characters are called \textbf{wildcards}
  \item using wildcards allow you to select filenames based on patterns of characters
  \item these wildcards are similar to some regular expression characters
\ei
\end{frame}

%------------------------------------------------

\begin{frame}
\frametitle{Wildcards}
{\small
\begin{center}
 \begin{tabular}{l l}
  \hline
   wildcard & description \\
  \hline
  \code{*} & matches any characters \\
  \code{?} & matches any single character \\
  \code{[characters]} & matches any character that is \\
   & a member of the set \textit{characetrs} \\  
  \code{[!characters]} & matches any character that is not \\
   & a member of the set \textit{characters} \\  
  \code{[[:class:]]} & matches any character that is \\
   & a member of the specified \textit{class} \\
  \hline
 \end{tabular}
\end{center}
}
\end{frame}

%------------------------------------------------

\begin{frame}[fragile]
\frametitle{Example}

Create a directory \code{dummy}, \code{cd} to it, and then create empty files:
\begin{knitrout}\footnotesize
\definecolor{shadecolor}{rgb}{0.969, 0.969, 0.969}\color{fgcolor}\begin{kframe}
\begin{alltt}
$ mkdir dummy
$ cd dummy
$ touch AGing.txt Bing.xt Gagging.text Going.nxt ing.ext
$ ls 
\end{alltt}
\end{kframe}
\end{knitrout}

\end{frame}

%------------------------------------------------

\begin{frame}[fragile]
\frametitle{\code{*} Wildcard}

Use {\hilit \code{*}} to refer to multiple files at once; it stands for \textit{anything}
\begin{knitrout}\footnotesize
\definecolor{shadecolor}{rgb}{0.969, 0.969, 0.969}\color{fgcolor}\begin{kframe}
\begin{alltt}
$ ls
  AGing.txt   Bing.xt
  Gagging.text Going.nxt ing.ext

$ ls G*
  Gagging.txt Going.nxt

$ ls *.xt
  Bing.xt
\end{alltt}
\end{kframe}
\end{knitrout}

\end{frame}

%------------------------------------------------

\begin{frame}[fragile]
\frametitle{\code{?} Wildcard}

The question mark {\hilit \code{?}} represents a \textit{single} character
\begin{knitrout}\footnotesize
\definecolor{shadecolor}{rgb}{0.969, 0.969, 0.969}\color{fgcolor}\begin{kframe}
\begin{alltt}
$ ls
  AGing.txt   Bing.xt
  Gagging.text Going.nxt ing.ext

$ ls ?ing.xt
  Bing.xt
\end{alltt}
\end{kframe}
\end{knitrout}

\end{frame}

%------------------------------------------------

\begin{frame}[fragile]
\frametitle{\code{[]} Wildcard}

Brackets {\hilit \code{[]}} can be replaced by whatever characters are within those characters:
\begin{knitrout}\footnotesize
\definecolor{shadecolor}{rgb}{0.969, 0.969, 0.969}\color{fgcolor}\begin{kframe}
\begin{alltt}
$ ls
  AGing.txt   Bing.xt
  Gagging.text Going.nxt ing.ext

$ ls [B]ing.*
  Bing.xt

$ ls [A-G]ing.*
  Bing.xt
\end{alltt}
\end{kframe}
\end{knitrout}

\end{frame}

%------------------------------------------------

\begin{frame}[fragile]
\frametitle{Combining Wildcards}

Wildcards can be combined:
\begin{knitrout}\footnotesize
\definecolor{shadecolor}{rgb}{0.969, 0.969, 0.969}\color{fgcolor}\begin{kframe}
\begin{alltt}
$ ls
  AGing.txt   Bing.xt
  Gagging.text Going.nxt ing.ext

$ ls *G*
  AGing.txt Gagging.txt Going.nxt

$ ls *i*.*e*
  Gagging.text ing.ext
\end{alltt}
\end{kframe}
\end{knitrout}

\end{frame}

%------------------------------------------------

\begin{frame}[fragile]
\frametitle{Test Yourself}
\begin{knitrout}\footnotesize
\definecolor{shadecolor}{rgb}{0.969, 0.969, 0.969}\color{fgcolor}\begin{kframe}
\begin{alltt}
  AGing.txt   Bing.xt
  Gagging.text Going.nxt ing.ext
\end{alltt}
\end{kframe}
\end{knitrout}

What command produces the output above:
\bi
  \item[A)] \code{ls *ing.*xt}
  \item[B)] \code{ls ?ing.*xt}
  \item[C)] \code{ls ?ing.?xt}
  \item[D)] \code{ls ?ing.xt}
  \item[E)] \code{ls *ing.?xt}
\ei

\end{frame}

%------------------------------------------------

\begin{frame}[fragile]
\frametitle{Test Yourself}
\begin{knitrout}\footnotesize
\definecolor{shadecolor}{rgb}{0.969, 0.969, 0.969}\color{fgcolor}\begin{kframe}
\begin{alltt}
  AGing.txt Going.nxt ing.ext
\end{alltt}
\end{kframe}
\end{knitrout}

What command produces the output above:
\bi
  \item[A)] \code{ls *ing.*xt}
  \item[B)] \code{ls ?ing.*xt}
  \item[C)] \code{ls ?ing.?xt}
  \item[D)] \code{ls ?ing.xt}
  \item[E)] \code{ls *ing.?xt}
\ei

\end{frame}

%------------------------------------------------

\begin{frame}
\frametitle{Wildcard Examples}
{\small
\begin{center}
 \begin{tabular}{l l}
  \hline
   Pattern & Matches \\
  \hline
  \code{*} & all files \\
  \code{a*} & any file beginning with ``a'' \\
  \code{*.txt} & any file ending with \code{.txt} \\
  \code{b*.txt} & any file beginning with ``b'' followed \\
   & by any characters and ending with ``.txt'' \\
   \code{[gst]*} & any file beginning with either \\
   & a ``g'', and ``s'', or a ``t'' \\
  \code{[[:digit:]]*} & any file beginning with a number \\
  \code{[[:upper:]]*} & any file beginning with an uppercase letter \\
  \hline
 \end{tabular}
\end{center}
}
\end{frame}

%------------------------------------------------

\begin{frame}
\begin{center}
\Huge{\hilit{Standard Input and Output}}
\end{center}
\end{frame}

%------------------------------------------------

\begin{frame}
\frametitle{}

\begin{center}
\Large{Many commands accept {\hilit input} \\ 
and produce {\mdlit output}}
\end{center}

\end{frame}

%------------------------------------------------

\begin{frame}
\frametitle{Input}

Input can come from:
\bbi
  \item the keyboard (a.k.a. \textbf{standard input})
  \item other files
  \item other commands
\ei

\end{frame}

%------------------------------------------------

\begin{frame}
\frametitle{Output}

Output can be:
\bbi
  \item printed on screen 
  \bi
    \item the command's results (a.k.a. \textbf{standard output})
    \item the status and error messages (a.k.a. \textbf{standard error})
  \ei
  \item written to files
  \item sent to other commands
\ei

\end{frame}

%------------------------------------------------

\begin{frame}
\frametitle{Output of commands}

\bi
  \item Consider the command \code{ls}
  \item \code{ls} sends the results to a special file called: \textit{standard output} or \textbf{stdout}
  \item \code{ls} sends status messages to another file called \textit{standard error} or \textbf{stderr}
  \item By default both \textit{stdout} and \textit{stderr} are linked to the screen and not saved into a disk file
\ei

\end{frame}

%------------------------------------------------

\begin{frame}
\frametitle{SI and SO}

\bbi
  \item The ``standard input'' is usually your keyboard
  \item The ``standard output'' is usually your terminal (monitor)
  \item But we can also redirect inputs and outputs
  \item I/O redirection allows us to change where output goes and where input comes from
  \item I/O redirection is done via the {\hilit \code{>}} redirection operator
\ei

\end{frame}

%------------------------------------------------

\begin{frame}
\begin{center}
\Huge{\hilit{Redirection Operator \code{>}}}
\end{center}
\end{frame}

%------------------------------------------------

\begin{frame}[fragile]
\frametitle{The \code{>} operator}

We can tell the shell to send the output of the \code{ls} command to the file \code{ls-output.txt}
\begin{knitrout}\footnotesize
\definecolor{shadecolor}{rgb}{0.969, 0.969, 0.969}\color{fgcolor}\begin{kframe}
\begin{alltt}
ls -l ~/Documents > ls-output.txt
\end{alltt}
\end{kframe}
\end{knitrout}

\end{frame}

%------------------------------------------------

\begin{frame}[fragile]
\frametitle{The \code{>>} operator}

We can tell the shell to send the output of the \code{ls} command and append it to the file \code{ls-output.txt}
\begin{knitrout}\footnotesize
\definecolor{shadecolor}{rgb}{0.969, 0.969, 0.969}\color{fgcolor}\begin{kframe}
\begin{alltt}
ls -l ~/Desktop >> ls-output.txt
\end{alltt}
\end{kframe}
\end{knitrout}

The contents in Desktop are appended to the file \code{ls-output.txt}

\end{frame}

%------------------------------------------------

\begin{frame}
\frametitle{Redirection}

\bbi
  \item[] {\hilit \code{>}} redirects STDOUT to a file
  \item[] {\hilit \code{<}} redirects STDIN from a file
  \item[] {\hilit \code{>>}} redirects STDOUT to a file, but appends rather than overwrites
  \item[] There is also {\hilit \code{<<}} but its use is more advanced than what we'll cover
\ei

\end{frame}

%------------------------------------------------

\begin{frame}
\frametitle{About Redirection}

\bbi
  \item Many times it is useful to send the output of a program to a file rather than to the screen
  \item Redirecting output to files is very common when extracting and combining data (think of merge!)
  \item Think of the redirection operator \code{">"} as an arrow that is pointing to where the output should go
\ei

\end{frame}

%------------------------------------------------

\begin{frame}[fragile]
\frametitle{Joining files with \code{cat}}

We can use {\hilit \code{cat}} and {\hilit \code{>}} to join two or more files:
\begin{knitrout}\footnotesize
\definecolor{shadecolor}{rgb}{0.969, 0.969, 0.969}\color{fgcolor}\begin{kframe}
\begin{alltt}
\hlcom{# remember the files from HW5?}
\hlcom{# (nflweather1960s.csv, ..., nflweather2010s.csv)}
ls nflweather*s.csv

\hlcom{# joining all the decades files in one single file}
cat nflweather*s.csv > allnfl.csv
\end{alltt}
\end{kframe}
\end{knitrout}

{\footnotesize The only issue here is that you would have appended column names}

\end{frame}

%------------------------------------------------

\begin{frame}
\frametitle{Joining files with \code{cat}}

Think about all the steps you would need to join the nfl-weather files without using the command line:
\bi
  \item You would have to open each file
  \item Open a new file \code{allnfl.csv}
  \item Start copy-pasting each adtaset into \code{allnfl.csv}
  \item Close all the decades files
  \item Save and close \code{allnfl.csv}
\ei

\end{frame}

%------------------------------------------------

\begin{frame}
\begin{center}
\Huge{\hilit{Redirection with pipes}}
\end{center}
\end{frame}

%------------------------------------------------

\begin{frame}
\frametitle{Redirection}

\bbi
  \item The idea behind pipes is that rather than redirecting output to a file, we redirect it into another command
  \item STDOUT of one command is used as STDIN to another command
  \item We can redirect inputs and outputs
  \item Redirection is done via the {\hilit \code{|}} pipe operator
\ei

\end{frame}

%------------------------------------------------

\begin{frame}[fragile]
\frametitle{Pipe example}

Let's say you want to count the number of \code{.csv} files in a specfic directory:
\begin{knitrout}\footnotesize
\definecolor{shadecolor}{rgb}{0.969, 0.969, 0.969}\color{fgcolor}\begin{kframe}
\begin{alltt}
\hlcom{# list csv files (one per line)}
\hlstd{ls} \hlopt{-}\hlnum{1} \hlopt{*}\hlstd{.csv}

\hlcom{# piping to count lines with 'wc -l'}
\hlcom{# (how many lines)}
\hlstd{ls} \hlopt{-}\hlnum{1} \hlopt{*}\hlstd{.csv} \hlopt{|} \hlstd{wc} \hlopt{-}\hlstd{l}
\end{alltt}
\end{kframe}
\end{knitrout}

The output of \code{ls -1} is piped to \code{wc -l}

\end{frame}

%------------------------------------------------

\begin{frame}[fragile]
\frametitle{Pipe example}

Let's say you want to inspect the contents of \code{/usr/bin}
\begin{knitrout}\footnotesize
\definecolor{shadecolor}{rgb}{0.969, 0.969, 0.969}\color{fgcolor}\begin{kframe}
\begin{alltt}
\hlcom{# long list of contents}
\hlstd{ls} \hlopt{/}\hlstd{usr}\hlopt{/}\hlstd{bin}

\hlcom{# using 'less' as a pager to see all the contents}
\hlstd{ls} \hlopt{/}\hlstd{usr}\hlopt{/}\hlstd{bin} \hlopt{|} \hlstd{less}
\end{alltt}
\end{kframe}
\end{knitrout}

The output of \code{ls} is piped to \code{less}

\end{frame}

%------------------------------------------------

\begin{frame}
\begin{center}
\Huge{\hilit{Command \code{grep}}}
\end{center}
\end{frame}

%------------------------------------------------

\begin{frame}
\frametitle{Regular Expressions with \code{grep}}

\bbi
  \item We can work with some regular expressions in the command line
  \item For that purpose we use the command {\hilit \code{grep}}
  \item \code{grep} can be very helpful for extracting particular rows from a file
\ei

\end{frame}

%------------------------------------------------

\begin{frame}[fragile]
\frametitle{\code{grep} example}

Consider the data \code{nflweather.csv}
\begin{knitrout}\footnotesize
\definecolor{shadecolor}{rgb}{0.969, 0.969, 0.969}\color{fgcolor}\begin{kframe}
\begin{alltt}
\hlcom{# rows containing Oakland (Raiders)}
grep \hlstr{'Oakland'} nflweather.csv

\hlcom{# rows from 2013}
grep \hlstr{'2013'} nflweather.csv
\end{alltt}
\end{kframe}
\end{knitrout}

\end{frame}

%------------------------------------------------

\begin{frame}[fragile]
\frametitle{\code{grep} example}

Consider the raw data \code{weather\_20131231.csv}
\begin{knitrout}\footnotesize
\definecolor{shadecolor}{rgb}{0.969, 0.969, 0.969}\color{fgcolor}\begin{kframe}
\begin{alltt}
\hlcom{# how many games in 2013}
grep \hlstr{'2013'} weather_20131231.csv | wc -l

\hlcom{# how many games in October 2013}
grep \hlstr{'10/[0-9]*/2013'} weather_20131231.csv | wc -l
\end{alltt}
\end{kframe}
\end{knitrout}

\end{frame}

%------------------------------------------------

\begin{frame}
\begin{center}
\Huge{\hilit{Command \code{curl}}}
\end{center}
\end{frame}

%------------------------------------------------

\begin{frame}
\frametitle{Command \code{curl}}

\bbi
  \item {\hilit \code{curl}} allows you to retrieve content from the Web
  \item \code{curl} stands for ``see URL''
  \item It access Internet files on your behalf, downling the content without any need of a browser window
\ei

\end{frame}

%------------------------------------------------

\begin{frame}[fragile]
\frametitle{\code{curl} example}

\begin{knitrout}\scriptsize
\definecolor{shadecolor}{rgb}{0.969, 0.969, 0.969}\color{fgcolor}\begin{kframe}
\begin{alltt}
\hlcom{# get the content of a URL}
curl \hlstr{"http://www.stat.berkeley.edu/~nolan/data/stat133/Saratoga.txt"}
\end{alltt}
\end{kframe}
\end{knitrout}

\end{frame}

%------------------------------------------------

\begin{frame}[fragile]
\frametitle{\code{curl} example}

\begin{knitrout}\scriptsize
\definecolor{shadecolor}{rgb}{0.969, 0.969, 0.969}\color{fgcolor}\begin{kframe}
\begin{alltt}
\hlcom{# get the content of a URL and save it to a file}
curl \hlstr{"http://www.stat.berkeley.edu/~nolan/data/stat133/Saratoga.txt"} 
-o saratoga.txt


\hlcom{# equivalently}
curl \hlstr{"http://www.stat.berkeley.edu/~nolan/data/stat133/Saratoga.txt"} 
> saratoga.txt
\end{alltt}
\end{kframe}
\end{knitrout}

\end{frame}

%------------------------------------------------

\begin{frame}
\begin{center}
\Huge{\hilit{Overview}}
\end{center}
\end{frame}

%------------------------------------------------

\begin{frame}
\frametitle{What good is it?}

\bbi
  \item Do I really need to learn these commands?
  \item The GUI file finder can do most of what we've seen (e.g. \code{ls}, \code{cd}, \code{mkdir}, \code{rmdir})
  \item Maybe it can't do what \code{cut} can do, but so what?
\ei

\end{frame}

%------------------------------------------------

\begin{frame}
\frametitle{Advantages of shell commands}

\bbi
  \item Shell commands gives us a programatic way to work with files and processes
  \item They allow you to \textbf{record} what you did
  \item They allow you to repeat it another time
  \item Volumne: Have many many operations to perform
  \item Speed: need to perform things quickly
  \item Less error prone: want to reduce mistakes
\ei

\end{frame}

%------------------------------------------------

\begin{frame}
\begin{center}
\Huge{\hilit{Command \code{cut}}}
\end{center}
\end{frame}

%------------------------------------------------

\begin{frame}
\frametitle{Command \code{cut}}

\bbi
  \item {\hilit \code{cut}} is most often used to extract columns of data from a field-delimited file
  \item They allow you to \textbf{record} what you did
  \item They allow you to repeat it another time
\ei

\end{frame}

%------------------------------------------------

\begin{frame}[fragile]
\frametitle{\code{cut} example}

\begin{knitrout}\scriptsize
\definecolor{shadecolor}{rgb}{0.969, 0.969, 0.969}\color{fgcolor}\begin{kframe}
\begin{alltt}
\hlcom{# 2nd column of a tab-separated file}
cut -f 2 starwarstoy.tsv


\hlcom{# 2nd column of a comma-separated file}
cut -f 2 -d \hlstr{","} starwarstoy.csv
\end{alltt}
\end{kframe}
\end{knitrout}

\end{frame}

%------------------------------------------------

\begin{frame}[fragile]
\frametitle{\code{cut} example}

\begin{knitrout}\scriptsize
\definecolor{shadecolor}{rgb}{0.969, 0.969, 0.969}\color{fgcolor}\begin{kframe}
\begin{alltt}
\hlcom{# columns 2-4 of a tab-separated file}
cut -f 2-4 starwarstoy.tsv


\hlcom{# columns 4-6 of a comma-separated file}
cut -f 4-6 -d \hlstr{","} starwarstoy.csv
\end{alltt}
\end{kframe}
\end{knitrout}

\end{frame}

%------------------------------------------------

\begin{frame}[fragile]
\frametitle{\code{cut} example}

\begin{knitrout}\scriptsize
\definecolor{shadecolor}{rgb}{0.969, 0.969, 0.969}\color{fgcolor}\begin{kframe}
\begin{alltt}
\hlcom{# columns 2-3 of first 10 rows in nflweather}
head -n 10 nflweather.csv | cut -f 2-4 
\end{alltt}
\end{kframe}
\end{knitrout}

\end{frame}

%------------------------------------------------

\end{document}
